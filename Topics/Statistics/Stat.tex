\documentclass{article}

\usepackage{hyperref}
\usepackage{amsmath}
\usepackage{amsfonts}

\DeclareMathOperator*{\argmax}{argmax}
\DeclareMathOperator*{\argmin}{argmin}
\newcommand{\dd}[1]{\mathrm{d}#1}


\title{Statistics}
\author{Jyotirmoy Banerjee}
\begin{document}
\maketitle

\section{Statistical test and p-value}
The p-value is the probability of seeing the effect (E) when the null hypothesis is true.
\[
\text{p-value} = P(E|H0)
\]
The null hypothesis (H0) assumes there is `no effect' or `relationship' by default. The alternate hypothesis (HA) is always framed to negate the null hypothesis. So, when the p-value is low enough, we reject the null hypothesis and conclude the observed effect holds.

\textbf{Alpha level:}
It is the cutoff probability for p-value to establish statistical significance for a given hypothesis test.
For an observed effect to be considered as statistically significant, the p-value of the test should be lower than the pre-decided alpha value.
Typically for most statistical tests, alpha is set as 0.05.

\textbf{Example:}
\begin{itemize}
\item Null Hypothesis (H0): The slope of the line of best fit (a.k.a beta coefficient) is zero.
\item Alternate Hypothesis (HA): The beta coefficient is not zero.
\end{itemize}

\end{document}

