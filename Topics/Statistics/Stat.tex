\documentclass{article}

\usepackage{hyperref}
\usepackage{amsmath}
\usepackage{amsfonts}

\DeclareMathOperator*{\argmax}{argmax}
\DeclareMathOperator*{\argmin}{argmin}
\newcommand{\dd}[1]{\mathrm{d}#1}


\title{Statistics}
\author{Jyotirmoy Banerjee}
\begin{document}
\maketitle

\section{Statistical test and p-value}
The null hypothesis (H0) assumes there is `no effect' or `relationship' by default. The alternate hypothesis (HA) is always framed to negate the null hypothesis. The p-value is the probability of seeing the effect (E) when the null hypothesis is true.
\[
\text{p-value} = P(E|H0)
\]
So, when the p-value is low enough, we reject the null hypothesis and conclude the observed effect holds.

\textbf{Alpha level:}
It is the cutoff probability for p-value to establish statistical significance for a given hypothesis test.
For an observed effect to be considered as statistically significant, the p-value of the test should be lower than the pre-decided alpha value.

Typically for most statistical tests, alpha is set as 0.05. 
What happens if it is say, 0.051? It is still considered as not significant. We do not call it as a weak statistical significant. It is either black or white. There is no gray with respect to statistical significance.
As a thumb rule, alpha is never set greater than 0.1. When the occurrence of the event is rare, we set a very low alpha. The rarer it is, the lower the alpha.

\textbf{Example:}
\vspace{-\topsep}
\begin{itemize}
\itemsep0em
\item Null Hypothesis (H0): The slope of the line of best fit (a.k.a beta coefficient) is zero.
\item Alternate Hypothesis (HA): The beta coefficient is not zero.
\end{itemize}
\vspace{-\topsep}

\section{Statistical Significance in A/B testing}
To make sure that we do not evaluate an experiment based on random results, we estimate the statistical significance - which is calculated by using the p-value.
P-value is created to show the exact probability that the outcome of the A/B test is a result of chance.
\vspace{-\topsep}
\begin{itemize} 
\itemsep0em
\item If the p-value is low ($<\alpha$) then we can say that version B is indeed better than version A.
\item If the p-value is high ($>=\alpha$) then our result could have happened randomly.
\end{itemize}
\vspace{-\topsep}

\end{document}

